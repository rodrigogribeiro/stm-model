\documentclass[a4paper, 12pt]{article}

\usepackage[portuges]{babel}
\usepackage[utf8]{inputenc}
\usepackage{amsmath}
\usepackage{indentfirst}
\usepackage{graphicx,hyperref}
\usepackage{multicol,lipsum}

\begin{document}
%\maketitle

\begin{titlepage}
    \begin{center}
    
    %\begin{figure}[!ht]
    %\centering
    %\includegraphics[width=2cm]{c:/ufba.jpg}
    %\end{figure}

        \Huge{Universidade Federal de Pelotas}\\
        \large{Departamento de Computação}\\ 
        \large{Programa de Pós-graduação em Ciência da Computação}\\ 
        \vspace{15pt}
        \vspace{95pt}
        \textbf{\LARGE{Relatório Final do Bolsista}}\\
        %\title{Especificação e Verificação Formal de Propriedades de Memórias Transacionais}
        \vspace{3,5cm}
    \end{center}
    
    \begin{center}
        %\begin{tabbing}
            Bolsista: Dr. Rodrigo Geraldo Ribeiro\\
            Supervisor: Dr. André Rauber Du Bois\\
         % \end{tabbing}
        \end{center}
    \vspace{1cm}
    
    \begin{center}
        \vspace{\fill}
             Pelotas\\
             2017
     \end{center}
\end{titlepage}
%%%%%%%%%%%%%%%%%%%%%%%%%%%%%%%%%%%%%%%%%%%%%%%%%%%%%%%%%%%

% % % % % % % % %FOLHA DE ROSTO % % % % % % % % % %

\begin{titlepage}
    \begin{center}
    
        \Huge{Universidade Federal de Pelotas}\\
        \large{Departamento de Computação}\\ 
        \large{Programa de Pós-graduação em Ciência da Computação}\\ 
\vspace{15pt}
        
        \vspace{85pt}
        
        \textbf{\LARGE{Relatório Final do Bolsista}}
        \title{\large{Especificação e Verificação Formal de Propriedades de Memórias Transacionais}}
    \end{center}
\vspace{1,5cm}
    
    \begin{flushright}

   \begin{list}{}{
      \setlength{\leftmargin}{4.5cm}
      \setlength{\rightmargin}{0cm}
      \setlength{\labelwidth}{0pt}
      \setlength{\labelsep}{\leftmargin}}

      \item Relatório Final de Pós-doutoramento.

      \begin{list}{}{
      \setlength{\leftmargin}{0cm}
      \setlength{\rightmargin}{0cm}
      \setlength{\labelwidth}{0pt}
      \setlength{\labelsep}{\leftmargin}}

            \item Bolsista: Dr. Rodrigo Geraldo Ribeiro\
            \item Supervisor: Dr. André Rauber Du Bois\
      \end{list}
   \end{list}
\end{flushright}
\vspace{1cm}
\begin{center}
  \vspace{\fill}
  01 de Setembro de 2017\\
\end{center}
\end{titlepage}
\newpage
% % % % % % % % % % % % % % % % % % % % % % % % % %
\newpage
\tableofcontents
\thispagestyle{empty}

\newpage
\pagenumbering{arabic}
% % % % % % % % % % % % % % % % % % % % % % % % % % %
\section{Resumo}

O uso de lógica formal em ciência da computação permite a especificação concisa
de sistemas e a verificação de suas propriedades. Durante o estágio pós-doutoral
o bolsista teve a oportunidade de aplicar lógica formal em problemas de
diferentes domínios: especificação de semânticas para memórias transacionais,
verificação formal de codificação binária para árvores de parsing e a formalização
de uma semântica denotacional para uma linguagem de domínio específico para
programação de padrões de bateria. Além do desenvolvimento dos trabalhos
anteriores, o bolsista colaborou em trabalhos com pesquisadores da UFMG e UDESC
no desenvolvimento de novos algoritmos para inferência de tipos para a linguagem
Haskell e na formalização de um algoritmo para
reconstrução de programas C incompletos. Os resultados obtidos em cada uma
destas atividades é descrito nos artigos abaixo listados e incluídos como anexos:

\begin{itemize}
  \item A Property Based Testing Approach for Software Transactional Memory
    Safety, a ser submetido para o periódico Science of Computer Programming.
  \item Certified Bit-coded Regular Expression Parsing, aceito para publicação
    no XXI Simpósio Brasileiro de Linguagens de Programação.
  \item A Domain Specific Language For Drum Beat Programming, aceito para
    publicação no XVI Simpósio Brasileiro de Computação Musical.
  \item Inference of Static Semantics for Incomplete C Programs, aceito para
    publicação no 45th ACM SIGPLAN Symposium on Principles of Programming
    Languages.
  \item Optional Type Classes for Haskell, em avaliação no periódico Science of
    Computer Programming.
  \item Type Inference for GADTs and Anti-unification, em avaliação no periódico
    Science of Computer Programming.
\end{itemize}

Além dos trabalhos acima citados, o bolsista ministrou um curso de 12 horas
entitulado ``Uma introdução ao assistente de provas Coq'' para
alunos de graduação e pós-graduação em ciência da computação da UFPel. O
material desenvolvido encontra-se disponível no seguinte endereço eletrônico:
\begin{center}
  \url{https://rodrigogribeiro.github.io/coqcourse/}
\end{center}
e participou da DeepSpec Summer School on verified algorithms de 13/08/2017 a
29/08/2017 na Universidade da Pensilvânia, Filadélfia, Estados Unidos.

\section{Conclusão e Trabalhos Futuros}

Durante o estágio pós-doutoral o bolsista teve a oportunidade de aplicar lógica
formal para a elaboração de trabalhos, envolvendo semântica de
linguagens de programação e verificação formal, que resultaram na publicação de
3 trabalhos em conferências, na submissão de 2 artigos para periódicos. 

Possíveis trabalhos futuros envolvem:

\begin{itemize}
  \item Formalização, em um assistente de provas, da linguagem de domínio
    específico HDrum, para programação de padrões de bateria. Atualmente, o
    bolsista já formalizou parte da linguagem de domínio específico utilizando o
    assistente de provas Coq.
  \item Formalização, em um assistente de provas, do algoritmo utilizado para
        reconstrução de programas C incompletos.
  \item Finalizar e submeter o artigo sobre a aplicação de teste baseado em propriedades
    para validar propriedades de segurança de memórias transacionais.

\vspace{5cm}

\begin{center}
  \begin{tabular}{ccc}
    Dr. Rodrigo Geraldo Ribeiro & \hspace{2cm} & Dr. André Rauber Du Bois \\
    Bolsista                    &              & Supervisor
  \end{tabular}
\end{center}
\newpage    
\section*{Anexos}

\end{itemize}
\end{document}